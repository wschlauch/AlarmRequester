\documentclass{scrartcl}

\usepackage[T1]{fontenc}
\usepackage[utf8]{inputenc}

\usepackage{booktabs}
\usepackage{amssymb}
\begin{document}
\title{Fortschrittsbericht für die Abfrage der Datenbank}
\maketitle
Die bisherige Implementierung basiert auf Ideen von möglichen KPIs welche von Prof.~Dr.~Rainer Röhrig auf dem Konsortial-Treffen der AlarmRedx Gruppe am 21.10.2016 in Stuttgart-Böblingen genannt wurden. Unter anderem erwähnte der die in der Tabelle~\ref{tab:alarms} aufgeführten KPIs. Dr.~Dirk Hueske-Kraus erwähnte, dass er sich auch ebenso gut eine Option für den Vergleich von Stationen vorstellen könne, welche auf den KPIs basiert.

Die erste Phase der Implementierung bestand on der Patient-Query implementierung, in welcher alle Antworten konsolen-basiert zurück gegeben wurden ohne spezielle Formatierung. Weiterhin war da die Idee des Vergleiches verschiedener Stationen anhand von KPIs noch nicht in den Raum gestellt worden.

In der zweiten Phase, Station-Query und Average-Query, kam dann von Dr.~Hueske-Kraus die Idee, dass man Stationen mit einander vergleichen können sollte. Diese Idee wurde aufgegriffen indem Überblicke über Durchschnittswerte einer Station bezüglich der KPI zurückgeliefert werden. Diese Option war auch ausschlaggebend für die Implementierung der HTML-Idee, d.h., die Antwort wird momentan als eine Tabelle mit den KPI zurück gegeben.

In der dritten Phase (aktuell) wird darüber nachgedacht und damit experimentiert, wie man diese Daten visualisieren kann/könnte, um ein auch für jemanden der Zahlen nicht mag und neu im Beruf ist/einen längerfristigen Überblick benötigt ein klares und schnell verständliches Bild der Station oder von einzelnen Patienten abzugeben.
\begin{table}
\centering
\begin{tabular}{p{5cm}cccc} \toprule
KPI vs. Querytool & Patient-Query & Station-Query & Average-Query \\ \midrule
\# Alarme & \checkmark & \checkmark & \checkmark \\ \midrule							 
\# Alarme/Kategorie 	 & \checkmark & \checkmark & \checkmark \\ \midrule
\# Situationen & ? & ? & ? \\	 \midrule		
Therapieanpassung & \checkmark & & \checkmark \\	\midrule					
Gleichzeitige Alarme & \checkmark & & \checkmark \\	\midrule				
generell Dauer bis zur Quittierung & \checkmark & & \\ \midrule			
Dauer bis zur Quittierung spezifischer Alarme	& \checkmark (only time) & & \\ \midrule		
Alarme bis zur Normalisierung &	\checkmark  & & \\	\midrule 
Dauer bis zur Normalisierung &	\checkmark & &  \\	\midrule	
Wo wurden Alarme quittiert & & & \\		\midrule	
Anzahl Alarme w\"{a}hrend der Stummschaltung & & & \\	\midrule
Implausible Alarme & & & \checkmark \\\midrule
temporale N\"{a}he (Zeitabst\"{a}nde definierbar) & & & \checkmark \\	\midrule
Reaktion auf Alarme	& & & \checkmark \\ \midrule
wurde der Patient behandelt oder war es willk\"{u}rliches Ende & & & \checkmark \\ \midrule					 
wurden die Grenzen angepasst & & & \checkmark \\ \bottomrule
\end{tabular}
\caption{Fortschritt der einzelnen Pakete.}
\label{tab:alarms}
\end{table}



\end{document}